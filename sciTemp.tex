\documentclass[preprint,tightenlines,eqsecnum,floats,aps,amsmath,amssymb,nofootinbib,prd,showpacs]{revtex4}
%\documentclass[twocolumn,superscriptaddress,floats,prd,nofootinbib,showpacs]{revtex4}
%\documentclass[tightenlines,superscriptaddress,eqsecnum,floats,nofootinbib,showpacs]{revtex4}
\usepackage{mathtools}
\usepackage{multirow}
\usepackage{amssymb}
\usepackage{stmaryrd}
\usepackage{amsmath}
\usepackage{amsfonts}
\usepackage{mathrsfs}
\usepackage{CJK}
\usepackage{amsmath,amssymb,amsfonts}
\usepackage{graphicx}
\usepackage{appendix}
% =============================================================
% equations
\def\be{\begin{equation}}
\def\ee{\end{equation}}
\def\ba{\begin{eqnarray}}
\def\ea{\end{eqnarray}}
\def\nn{\nonumber}



\newcommand{\secref}[1]{Sec.~\ref{#1}}
%\newcommand{\lemref}[1]{Lemma~\ref{#1}}
\newcommand{\eqnref}[1]{(\ref{#1})}
\newcommand{\figref}[1]{Fig.~\ref{#1}}
\newcommand{\appref}[1]{Appendix~\ref{#1}}


% =============================================================



% =============================================================


% =============================================================
% define notations:
%\newcommand{\Pl}{\ell_\mathrm{Pl}} % Planck length
%\newcommand{\muzero}{\mu^o} % \mu naught
%\newcommand{\mubar}{{\bar \mu}} % mu bar
%\newcommand{\abs}[1]{{\left|{#1}\right|}} % abs (variant delimiters)
%\newcommand{\Abs}[1]{{\Big|{#1}\Big|}} % abs (fixed delimiters)
%\newcommand{\norm}[1]{{\Vert{#1}\Vert}} % norm
%\newcommand{\pinner}[2]{({#1},{#2})_\mathrm{phy}} % physical inner product
%\newcommand{\inner}[2]{{\langle {#1}\vert {#2} \rangle}} % inner product
%\newcommand{\ket}[1]{\vert{#1}\rangle} % ket
%\newcommand{\bra}[1]{\langle{#1}\vert} % bra
%\newcommand{\kt}{{\tilde{K}}} % K tilde
%\newcommand{\R}{\mathcal {R}} % 4-d curvature scalar
%\newcommand{\ct}{\tilde{c}}
%\newcommand{\SU}{\mathscr{S}\mathscr{U}} % SU(2) Lie algebra
%\newcommand{\ints}{{\int_\Sigma}} % integral
%\newcommand{\eff}{{eff}} % effective
%
%\newcommand{\sgn}{\mathrm{sgn}} % sign
%\newcommand{\Tr}{\mathrm{Tr}} % trace
%\newcommand{\grav}{\mathrm{gr}} % gravitational part
%\newcommand{\sca}{\mathrm{sc}} % scalar part
%\newcommand{\kin}{\mathrm{kin}} % kinematic (Hilbert space, states)
%\newcommand{\phy}{\mathrm{phy}} % physical (Hilbert space, states)
%\newcommand{\hil}{\mathcal{H}} % Hilbert space
%\newcommand{\Euc}{H^{E}} % Euclidean scalar
%
%% =============================================================
%% triad and co-triad:
%\newcommand{\ftriad}[2]{{}^o\! e^{#1}_{#2}} % fiducial co-triad
%\newcommand{\fcotriad}[2]{{}^o\!\omega_{#1}^{#2}} % fiducial co-triad
%\newcommand{\fq}{{}^o\!q} % fiducial 3-metric
%\newcommand{\tE}{\mbox{$\tilde{E}$}} % densitized triad




% ===============================================================


% ===============================================================
% ===============================================================
\begin{document}
%\date\today

\title{Suitable resolution for EOS tables in neutron star investigation}

\author{Houyuan Chen}
\affiliation{School of Physics and Optoelectronic Technology, South China University of
Technology, Guangzhou 510641, P.R. China}

\author{Dehua Wen\footnote{Corresponding author. wendehua@scut.edu.cn}}
\affiliation{School of Physics and Optoelectronic Technology, South China University of
Technology, Guangzhou 510641, P.R. China}

\author{Na Zhang}
\affiliation{School of Physics and Optoelectronic Technology, South China University of
Technology, Guangzhou 510641, P.R. China}







\date{\today}


\begin{abstract}
Inasmuch as the hydrostatic structure of interior neutron stars uniquely depends on the equation of state (EOS), the inverse constraints on EOS from astrophysical observation have been important methods to reveal the properties of the high-density matter. Up to date, most of EOSs for neutron-star matter are given in tabulations, but these numeric tables could be quite different in resolution. To guarantee both the accuracy and efficiency in computing Tolman-Oppenheimer-Volkoff (TOV) equation, a concise standard for generating EOS tables with suitable resolution is investigated in this work. Via massive instances and comparisons, it is shown that the EOS tables with 50 points logarithmic-uniformly located at supra-nuclear density segment ( $\rho_{0} \sim 10 \rho_{0} $, where $\rho_{0}$ is the nuclear saturation density) would correspond to the interpolation-induced errors at $\sim 0.02\%$ for gravitational mass $M$ and $\sim 0.2\%$ for tidal deformability $\Lambda$.

\end{abstract}

\pacs{97.60.Jd;04.40.Dg;26.60.Dd;26.60.Gj}


\maketitle


\section{Introduction}

The pulsars in observation are generally considered as neutron stars in theory. There is maximum mass ($M_{max}$) for a neutron star to delimit whether it is hydrostatic stable or it will finally collapse into a black hole through the oscillation process. Most of observed neutron stars have mass around 1.4 solar masses ($\rm{M}_{\odot}$), and PSR J0348+0432 with a mass of $2.01 \pm 0.04 \, \rm{M}_{\odot}$ is the heaviest one which has been accurately measured. In theoretical perspectives, the macroscopic properties of neutron stars depend strongly on the equation of state (EOS) of high-density matter. The current theoretical constraints on the $M_{max}$ are far from uniform. Recent constraints extracted from the observation of gravitational waves (GW170817) on the  maximum mass are $M_{max} < 2.17\rm{M}_{\odot}$, and on the radius of a canonical neutron star are $11.0 < R_{1.4}/ \rm{km} < 13.2$.

Lots of pioneering works have been done to predict the gravitational waves emitted from a binary neutron star (BNS) system. The numeric simulations of BNS mergers, which are various with different choices of the EOSs, have illustrated possibility for probing indirectly the properties of neutron-star matter from gravitational waves (GWs). Postnikov \emph{et al.} pointed out that the dimensionless tidal deformability $\Lambda$, which can be revealed from the GW signals during inspiral process of BNS, is capable of characterizing different EOSs. On August 17, 2017, the LIGO and Virgo observatories have made the first GW detection of BNS coalescence (GW170817), of which the combined tidal deformability was constrained to $\Lambda_{1.4}  < 800$ for canonical neutron stars from the first analysis. An improved analysis of the GW170817 provided both the upper and lower limits for the tidal deformability as ${\Lambda _{1.4}} = 190_{ - 120}^{ + 390}$, which leads to a constraint on the EOS at twice nuclear saturation density as $p(2{\rho _0}) = 21.85_{ - 10.61}^{ + 16.85} \, \rm{MeV/fm^3}$. With observational constraints on $\Lambda$, Most \emph{et al.} generated millions of EOSs from their parameterized sets and then exploited more than $10^{9}$ equilibrium models for neutron stars to measure the typical radius ${R_{1.4}} = 12.39_{ + 1.06}^{ - 0.39}$ $\textrm{km}$ at $2\sigma $ level.


The theoretical determination of $\Lambda  = (2/3){({c^2}/G)^5}{(R/M)^5}{k_2}$ requires precise inner solutions of the Tolman-Oppenheimer-Volkoff (TOV) equation. The relevant tidal Love number $k_2$ is determined by the hydrostatic distribution of the stars. The tidal deformability $\Lambda$ deduced from $k_2$ can be used to discriminate the EOSs, by contrasting to GW observation of BNS merger.
On the other hand, there is no unified model yet to describe the EOS of the compressed matters. Even in a specific model, it generally takes complex computation to provide the $\rho-\varepsilon-p$ relation, where $\rho$ is the baryon number density, $\varepsilon$ is the energy density and $p$ is the pressure. Therefore, numeric EOS tables become a convenient choice in neutron star study. For the realistic EOSs (that is, the tabular EOSs), the solutions of stellar structure have to be given by numerical integration.

In the integration process, the single-step errors must be restrained to provide accurate results of $k_2$. By contrast, different from the integration errors that can be handled simply with shorter step-sizes, the interpolation-induced errors are mainly affected by the resolution of EOS tables. Nonetheless, when employing a huge number of EOSs to investigate the neutron-star characteristics by Bayesian methods, the efficiency of the interpolation is of crucial important. The too large-size EOS tables are thus unpractical in these statistical studies. Therefore, the suitable resolution for EOS tables can be significant in neutron-star investigations.


This paper is organized as follows. In Sec.$\,$II, a brief introduction of some pragmatic techniques to deal with both the integration and the interpolation is given first. Then two widely-used interpolation methods are introduced to inspect the interpolation-induced errors from EOS tables with different resolution under the most common mesh. The minimal size of an EOS table to provide accurate results of $M$ and $\Lambda$ is also discussed in this section. In Sec.$\,$III, we extend the investigation onto different meshing methods for EOS tables to further examine the model-dependence of the discussion. At the end, a concise summary will be given.


\section{numeric setup and EOS grid resolution}

The TOV equation is the hydrostatic equilirium equation of general relativity to describe the internal structure of a static, non-rotating and spherical neutron star, which can be written as,
\begin{equation} \label{TOV}
\frac{{dp}}{{dr}} =  - \frac{{G(\varepsilon + \frac{{p}}{{{c^2}}})(m + \frac{{4\pi {r^3}p}}{{{c^2}}})}}{{{r^2}(1 - \frac{{2Gm}}{{r{c^2}}})}} \quad , \quad \frac{{dm}}{{dr}} = 4\pi \varepsilon {r^2}
\end{equation}
where  $m(r)$ refers to the gravitational mass within the radius $r$ measured by observer at infinity, $G$ and $c$ are gravitational constant and light speed respectively.

An important method to investigate macroscopic properties of neutron stars is to numerically integrate Eq.$\,$(\ref{TOV}) from the center, where $m=0$, $r=0$, $\varepsilon = \varepsilon_c$, to the surface, where $p=0$, $r=R$ and $m(R)=M$. The widely used forth-order Runge-Kutta (RK-4) method  is applied as the high precision integration algorithm in this work. Moreover, we denote the relative deviation, that a quantity $Q$ with respect to its exact value $Q_T$, as $resQ=\left| Q-Q_T \right|/Q_T$ to discuss the precision issue.

%\begin{figure}
%\centering
%\includegraphics [width=0.6\textwidth]{resP.eps}
%\caption{ \label{resP} The percentage local residuals for both ordinary and adaptive RK-4. The percentage local residuals are defined as the errors of the pressure increments in each step, divided by the precise values of pressure increments. The dash line is RK-4 computed with a fixed step-size $h=10\textrm{(m)}$ while the solid line applies adaptive step-size. The slant line pattern denotes the objective local residual as the adaptive reference. The central density of the two instances is $4 \rho_0$ and the EOS is APR.}
%\end{figure}

The RK-4 method with adaptive step-size control can guarantee the global accuracy of the radius. In a fixed step-size computation, the local errors at the outer layers can increase rapidly when integrating outwards, which have been shown in Fig.. The residuals for the final outcomes of radius are strongly relevant to these errors. The adaptive method effectively controls the local integration errors at outer crust layer until several meters to stellar surface. This technique is expected to improve computational precision significantly at crust which is of great concern in the investigations of low-mass stars. It is of crucial importance to control the radial deviation especially when the suitable step-size is not clear. With a proper method to integrate, the induced radial errors should be handled to $< 0.1\%$.

Apparently, the most important input in solving Eq.$\,$(\ref{TOV}) is the $\varepsilon - p$ relation. Currently, EOSs in tabulation form are most common to employ in neutron-star investigations. To properly use EOS tables, it is necessary to apply interpolation of monotonicity preserving to obtain the intermediate values. The errors generated from interpolation process could be quite different according to both the specific method of interpolation and the resolution of EOS table.

A simple but non-rigorous approximation for EOS table could be piecewise polytropic,
\begin{equation} \label{polytropicEOS}
{\log _{10}}p = {\log _{10}}K + \gamma \left( {{{\log }_{10}}\varepsilon  - {{\log }_{10}}{\varepsilon _0}} \right)
\end{equation}
where $\varepsilon_0$, $\gamma$ and $K$ are considered as constant within each segment.

Under the polytropic approximation, the simplest idea to interpolate could be that transform all the data points from EOS table into logarithmic space then to implement linear interpolation. For example, the value $(\varepsilon , p)$ between its nearest neighbors $(\varepsilon_n , p_n)$ and $(\varepsilon_{n+1} , p_{n+1})$ could be given as,
\begin{equation} \label{linearinterp}
\frac{{{{\log }_{10}}p - {{\log }_{10}}{p_n}}}{{{{\log }_{10}}\varepsilon  - {{\log }_{10}}{\varepsilon _n}}} = \frac{{{{\log }_{10}}{p_{n + 1}} - {{\log }_{10}}{p_n}}}{{{{\log }_{10}}{\varepsilon _{n + 1}} - {{\log }_{10}}{\varepsilon _n}}}
\end{equation}

In addition to linear interpolation, an advanced method that simultaneously preserves the monotony and the first derivative continuity of the EOS is the Piecewise-Cubic-Hermite-Interpolating-Polynomial (PCHIP) which could be written as follow,
\begin{eqnarray}
p\left( \varepsilon  \right) &=& {p_n}{\alpha _n}\left( \varepsilon  \right) + {p_{n + 1}}{\alpha _{n + 1}}\left( \varepsilon  \right) + {k_n}{\beta _n}\left( \varepsilon  \right) + {k_{n + 1}}{\beta _{n + 1}}\left( \varepsilon  \right),\label{pchipII}
\end{eqnarray}
where $k_{n},k_{n+1}$ are respectively slopes of interpolated function $p(\varepsilon)$ at $\varepsilon_n$ and $\varepsilon_{n+1}$. The specific rules to give the knot slopes $k_{n}$ are introduced in the appendix. The four Hermitian functions are given as,
\begin{eqnarray}
{\alpha _n}\left( \varepsilon  \right) &=& \left( {1 + 2\frac{{\varepsilon  - {\varepsilon _n}}}{{{\varepsilon _{n + 1}} - {\varepsilon _n}}}} \right){\left( {\frac{{\varepsilon  - {\varepsilon _{n + 1}}}}{{{\varepsilon _n} - {\varepsilon _{n + 1}}}}} \right)^2} ,\label{pchipI} \\
 \quad {\alpha _{n + 1}}\left( \varepsilon  \right) &=& \left( {1 + 2\frac{{\varepsilon  - {\varepsilon _{n + 1}}}}{{{\varepsilon _n} - {\varepsilon _{n + 1}}}}} \right){\left( {\frac{{\varepsilon  - {\varepsilon _n}}}{{{\varepsilon _n} - {\varepsilon _{n + 1}}}}} \right)^2},\\
{\beta _n}\left( \varepsilon  \right) &=& \left( {\varepsilon  - {\varepsilon _n}} \right){\left( {\frac{{\varepsilon  - {\varepsilon _{n + 1}}}}{{{\varepsilon _n} - {\varepsilon _{n + 1}}}}} \right)^2}, \\
\quad {\beta _{n + 1}}\left( \varepsilon  \right) &=& \left( {\varepsilon  - {\varepsilon _{n + 1}}} \right){\left( {\frac{{\varepsilon  - {\varepsilon _n}}}{{{\varepsilon _n} - {\varepsilon _{n + 1}}}}} \right)^2},
\end{eqnarray}


In order to facilitate the analysis, we divide the tabular EOSs into two parts: the low density segment $[0, \rho_0]$ and supra-nuclear segment $[\rho_0 , 10\rho_0]$. For comparison purposes, we define APR-a as an exact EOS example, which is spline-fitted from the well-known APR EOS using a smoothing parameter 0.989 (corresponding to a smallest R-square). To discuss the errors induced from interpolation process, we uniformly sample from APR-a at logarithmic density in the two segment to produce EOS tables in different resolution. The minimum density for the samplings in this paper is $10^4$ $\rm{kg/m^3}$. The grid points in each density segment are logarithmic-uniform as log$\rho_{n+1}$-log$\rho_n$=constant. Up to date, most of available EOS tables are provided with density grid points of this type (denoted as $U$-grid) at high density segment. We also use a concise symbol to denote the resolution of EOS table. For example, APR-20(200) indicates that the produced table of APR EOS contains totally 200 points of which 20 points are distributed at supra-nuclear density segment.

We sample from APR-a under $U$-grid to produce EOS tables of scale 20(200), 20(400), 50(200) and 50(100). Then we separately estimate the relative residuals in Tab.$\,$\ref{Umesh} with the two interpolation methods. The integration process is adequately precise in this table, where the mass errors are at $\sim0.001\%$ level which are negligible compared with the interpolation-induced errors.

\begin{table}[!htbp]
\centering
\begin{tabular}{|c|c|c|c|c|c|c|c|c|c|}
\hline
\multicolumn{2}{|c|}{ \multirow{2}*{Exact} } & \multicolumn{4}{c|}{APR-20(200)} & \multicolumn{4}{c|}{APR-50(200)}\\
\cline{3-10}
\multicolumn{2}{|c|}{} & \multicolumn{2}{c|}{Linear} & \multicolumn{2}{c|}{PCHIP} & \multicolumn{2}{c|}{Linear} & \multicolumn{2}{c|}{PCHIP}\\
\hline
 $\rho_c(\rho_0)$ & $M(M_{\odot})$ & $resM(\%)$ & $res\Lambda(\%)$ & $resM(\%)$ & $res\Lambda(\%)$ & $resM(\%)$ & $res\Lambda(\%)$ & $resM(\%)$ & $res\Lambda(\%)$\\
\hline
3.3377&1.0900 	&0.0669	&0.3517	&0.0887	&0.4766	&0.0186	&0.0624	&0.0135	&0.0525\\
4.0071&1.4000	&0.4114	&2.8483	&0.3859	&2.4134	&0.0458	&0.3946	&0.0284	&0.2269\\
4.6828&1.6500 	&0.2489	&1.4483	&0.1259	&0.6086	&0.0496	&0.4255	&0.0253	&0.2667\\
6.7543&2.0500 	&0.0562	&0.2848	&0.0104	&0.0797	&0.0232	&0.1820	&0.0108	&0.1404\\

\hline
\multicolumn{2}{|c|}{ \multirow{2}*{Exact} } & \multicolumn{4}{c|}{APR-20(400)} & \multicolumn{4}{c|}{APR-50(100)}\\
\cline{3-10}
\multicolumn{2}{|c|}{} & \multicolumn{2}{c|}{Linear} & \multicolumn{2}{c|}{PCHIP} & \multicolumn{2}{c|}{Linear} & \multicolumn{2}{c|}{PCHIP}\\
\hline
 $\rho_c(\rho_0)$ & $M(M_{\odot})$ & $resM(\%)$ & $res\Lambda(\%)$ & $resM(\%)$ & $res\Lambda(\%)$ & $resM(\%)$ & $res\Lambda(\%)$ & $resM(\%)$ & $res\Lambda(\%)$\\
\hline
3.3377&1.0900 	&0.0672	&0.3788	&0.0894	&0.4812	&0.0298	&0.8041	&0.0130	&0.3395\\
4.0071&1.4000	&0.4066	&2.8416	&0.3816	&2.4207	&0.0465	&0.4866	&0.0404	&0.3764\\
4.6828&1.6500 	&0.2313	&1.4434	&0.1251	&0.6026	&0.0537	&0.5554	&0.0408	&0.4020\\
6.7543&2.0500 	&0.0562	&0.2882	&0.0103	&0.0758	&0.0208	&0.2742	&0.0114	&0.0917\\

\hline
\end{tabular}
\caption{Relative residuals of applying the two interpolation methods. The most left column of $\rho_c$ is the central density for each corresponding row. The relative residuals $resM$ and $res\Lambda$ are defined as the relative deviations of $M$ and $\Lambda$ to their exact solutions.}\label{Umesh}
\end{table}

As we know, the mass results are most sensitive to the EOS of the star's central region. As shown in Tab.$\,$\ref{Umesh}, the overall smaller $resM$ indicate that PCHIP interpolation have advantages over linear interpolation. The precision of $\Lambda$ is affected simultaneously by $k_2$, $M$ and $R$. Under $U$-grid the precision of $k_2$ is generally on the same order of $M$ and their relative residuals can be estimated as rough views of the hydrostatic solution precision. The radial errors are mainly related to the integrational step-size which we will not discuss in detail, but as a conclusion the relative errors of $R$ is $\sim 0.01\%$ for the four EOS tables in Tab.$\,$\ref{Umesh}. Therefore, for APR-20(200) and APR-20(400), which assure $resR \ll resM$ and $resk_2 \sim resM$, the Love-number relative residuals are $\sim 6 \, resM$, according to the definition.

Our calculation shows that for both the PCHIP and linear interpolation, EOS tables with denser data-points at high density side would significantly decrease both the mass residuals and the Love number residuals. The data-point amount of APR-20(200) is the same as APR-50(200), however, APR-50(200) produces much more precise outcomes because of better resolution at high density side. On the other hand, through comparing the data of APR-20(400) and APR-20(200) in Tab.$\,$\ref{Umesh}, it is shown that the resolution improvement at low density  can not remarkably reduce the interpolation-induced errors.

The comparisons indicate that the resolution of EOS table at supra-nuclear density is much more important than the total amount of data-points. In addition, we notice that the too small-size EOS table APR-50(100) would violate the approximation that $resk_2 \sim resM$, rendering the drop of $\Lambda$ precision, especially for the low-mass stars. With some more trials on the resolution at the both density segments, the minimal scale for $U$-grid APR EOS is finally determined as 50(150) which correspond to the interpolation-induced errors $\sim 0.02\%$ for $M$ and $\sim0.2\%$ for $\Lambda$.

Actually, lots of EOSs in the literatures prefer to adopt the table scale around 20(200) under $U$-grid, such as SFHo, GShen and LS EOSs. The interpolation-induced errors for the stellar mass $M$ produced by these EOS tables are expected to be $0.1\% \sim 1\%$. According to above discussion, it is suggested that when we produce the EOS table, it is better to contain more than 50 grid points at supra-nuclear density segment ($\rho_0 \sim 10\rho_0$). That would significantly reduce the interpolation-induced errors and thus result in much more precise solutions for such as the stellar mass $M$ and tidal deformability $\Lambda$.


\section{Dependence on grid specification}

To eliminate accidental factors, we further inspect the interpolation errors under several different grid modes,
\begin{alignat}{2} \label{Def_grid}
& \frac{{{{\log }_{10}}{\rho _{n + 1}} - {{\log }_{10}}{\rho _n}}}{{{{\log }_{10}}{\rho _n} - {{\log }_{10}}{\rho _{n - 1}}}}= \begin{cases}  {C_1} , \\   {\frac{{1 + {e^{ - {C_2} \cdot (n + 1)}}}}{{1 + {e^{ - {C_2} \cdot n}}}}} , \end{cases}
& &
\begin{aligned}
\text{for } {G{\rm{ - grid}}} \\
\text{for } {ue{\rm{ - grid}}}
\end{aligned}
\end{alignat}
where $C_1$ and $C_2$ are adjustable coefficients to meet the resolution requirement. Apparently, taking $C_1=1$ for $G$-grid is equivalent to $U$-grid we mentioned in Sec.$\,$II.

We create the APR-50(150) EOS tables under the four different grid modes as the follows. (i)$\,$We separately take $C_1=1$ at low density and $C_1=0.1$ at high density to define $uu$-grid. (ii)$\,$We stipulate that $ue$-grid uses $C_2=0.56$ at supra-nuclear density segment but shares the same grids from $uu$-grid at low density segment. (iii)$\,$The EOS table under $U$-grid is logarithmic-uniform ($C_1=1$) that consistent with Sec.$\,$II. (iv)$\,$To generate the EOS table of scale 50(150) under $G$-grid, the constant should be $C_1=0.9785$ at the both density segments. A concise example of the four grids for APR-50(150) are plotted in Fig..

%\begin{figure}
%\centering
%\includegraphics [width=0.6\textwidth]{GridType.eps}
%\caption{ \label{GridType} Baryon density grid points for APR-50(150) under four different grid modes. The dash line denotes the saturation density $\rho_0$.}
%\end{figure}

The $uu$-grid means uniform distribution at $\rho_0 \sim 10\rho_0$ and logarithmic-uniform at $0 \sim \rho_0$. It is an extreme distribution that the grid points are concentrated excessively at the high density side. The $U$-grid could be considered as the opposite extreme with respect to the $uu$-grid. Any distribution that less-contractive than $U$-grid should be irrational according to the discussion in Sec. II that the grid points should concentrate at high density side. From the definition of $G$-grid and $ue$-grid, by contrast, they are almost transitional schemes between $U$-grid and $uu$-grid. The $ue$-grid, different from the frameworks of $G$-grid, is designed to restrain the contractive rate that the final interval of the logarithmic grid is expected to be about half of the beginning one.

%\begin{figure}
%\centering
%\includegraphics [width=0.6\textwidth]{Res4Mesh.eps}
%\caption{ \label{EOSdispMethod} Residuals comparison of APR EOS of scale 20(200) and 50(150), under four different mesh grids and with the PCHIP interpolation. The interpolation-induced errors are defined as the relative deviations of $M$ to the exact solutions. The horizontal axis is central density while the vertical dash lines denote the mass.}
%\end{figure}

According to the residuals comparison of the four meshing methods in Fig., it is obvious that the interpolation-induced errors are related to both the choice of the EOS mesh and the central density. At a given mass, the EOS tables that provide data-points more concentrated at vicinity of the central density would naturally correspond to smaller errors. For example, the $uu$-grid that most stressed on the high density side results in the most precise outcomes at $\sim 2 \, \rm{M}_{\odot}$ but the least precise outcomes at lower masses. Additionally, the interpolation errors of EOS itself are generally small nearby each data-point and achieving maximum at the intermediate position before the next point. These regular changes lead to oscillatory contours in Fig..

Considering both the practicability and the overall accuracy, the widely-used $U$-grid remains good choice to provide EOS tables. Although there is certain meshing-method dependence in accuracy issue, we may still conclude from Tab.$\,$\ref{Umesh} and Fig. that the EOS tables of scale 50(150) could effectively restrain the interpolation-induced errors of stellar mass to $resM\sim 0.02\%$ for the cases that $1.09 \, \rm{M}_{\odot}<M<2.05 \, \rm{M}_{\odot}$.

Some may doubt the universality of the conclusion because all the EOS tables in comparison are sampled from a single EOS model. To make the conclusion more reliable, we extend the same interpolation trials on the parameterized asymmetric nucleon matter EOSs. We generated several tens of EOS tables in the resolution of 50(150), and then separately estimated the interpolation errors with different methods. We found them consistent with the former analyses of APR EOS, that is, the interpolation errors of stellar masses $M$ are $\sim 0.02\%$ for intermediate-mass neutron stars.


\section{summary and conclusion}

We reinvestigated the traditional numeric methods for computing TOV equation including both integration techniques and interpolation methods, of which all the C codes have been open-source in. For the convenience of discussion, we separated the global errors into integrational errors and interpolation-induced errors. As the integration residuals nearby the stellar surface are divergent and the radial precision can be affected strongly by the choice of step-size, the adaptive step-size method is adopted to solve this problem.


Currently, a bulk of available EOS tables are provided in the general scale of $\sim20(200)$ . As the errors from integration process can be well-handled, the dominant errors of using these EOS tables would come from interpolation process. The relation between interpolation-induced errors and the EOS table resolution under $U$-grid are investigated in detail. It is concluded that the increasing amount of data-points at supra-nuclear density segment ($\rho_0 \sim 10 \rho_0$) could effectively reduce the interpolation-induced errors. The EOS table of scale 50(150) would correspond to the relative residuals for $M$ and $\Lambda$ to respectively $\sim0.02\%$ and $\sim0.2\%$, much more accurate than the 20(200) ones. In addition, it is also shown  that the PCHIP method is more accurate   in interpolating  EOS at neutron-star core than the linear method.

The dependence of the meshing methods is inspected. Among the four specified meshing methods, $U$-grid remains suitable method for generating EOS tables to compute intermediate-mass neutron stars. The EOS-model dependence is also examined by involving the parameterized asymmetric nucleonic matter EOSs . It is concluded that EOS tables of scale 50(150) could still significantly improve the accuracy compared with the general scale 20(200), despite of certain differences in meshing methods, EOS models or interpolation methods.

\begin{acknowledgements}
We would like to thank Bao-An Li for helpful discussions. This work is supported by the National Natural Science Foundation of China (No.11722546 and No.11275073), talent program of South China University of Technology (No. K5180470). This project is sponsored by CSC and has made use of NASA's Astrophysics Data System.

\end{acknowledgements}

%\newpage

\bibliographystyle{unsrt}
\bibliography{library.bib}


\appendix

\section{piecewise-cubic-hermite-interpolation}
Varying with the specific method to give the boundary slopes $k_n$ and $k_{n+1}$ at Eqs. (\ref{pchipI}-\ref{pchipII}), there are countless executions of Hermite interpolation, and the one used in this paper is a specified weighted average of differential slopes as its nodal slopes $k_n$ to preserve the shape of interpolated function from the origin data-points. The expression of nodal slopes $k_n$ for inner-points and endpoints are separately given in Eqs. (\ref{pchipinnerpoint})-(\ref{pchipendpoint}).

Generally, $k_{n}$ at each knot is uniquely determined by differential aspects of proximal points. Hermite interpolation therefore ensures the first derivative $dp/d\varepsilon$ is continuous everywhere. We denote the right differential step of $\varepsilon_n$ as $h_n=\varepsilon_{n+1} - \varepsilon_n$ and the right differential slope as $\nu_n=(p_{n+1}-p_n)/h_n$. When $sgn(\nu_{n-1}) \ne sgn(\nu_{n})$, we choose $k_n = 0$ so that the extremum of data-points could be coincident with that of interpolated function, although it is not very likely to have $sgn(\nu_{n-1}) \ne sgn(\nu_{n})$ within a rigorous EOS. In the general case, $k_n$ of the inner-points are given as,
\begin{equation} \label{pchipinnerpoint}
\begin{array}{l}
\widetilde {{\nu _n}} = \frac{{{h_{n - 1}}{\nu _{n - 1}} + {h_n}{\nu _n}}}{{{h_{n - 1}} + {h_n}}}\\
{k_n} = \frac{{3{\nu _{n - 1}}{\nu _n}}}{{{\nu _{n - 1}} + {\nu _n} + \widetilde {{\nu _n}}}}
\end{array}
\end{equation}

Slopes estimation at the two endpoints $k_1$ and $k_{end}$ are slightly different from that at inner-points. We denote the differential step from an endpoint to its nearest neighbour as $h_{1*}=\varepsilon_{1*}-\varepsilon_{1,end}$, and denote that from nearest neighbour to sub neighbour as $h_{2*}=\varepsilon_{2*}-\varepsilon_{1*}$, while the corresponding differential slopes are $\nu _{1*}=(p_{1*}-p_{1,end})/h_{1*}$ and $\nu _{2*}=(p_{2*}-p_{1*})/h_{2*}$. In the estimation, we firstly give $k_{1,end}$ as Eq.$\,$(\ref{pchipendpoint}). If $sgn(k_{1,end}) \ne sgn(\nu_{1*})$ we choose $k_{1,end}=0$, else, if $sgn(\nu_{1*}) \ne sgn(\nu_{2*})$ and $\left| k_{1,end} \right| > \left| 3\nu_{1*} \right|$ we choose $k_{1,end}=3\nu_{1*}$, and only when all the judgements above are false $k_{1,end}$ remains invariant.
\begin{equation} \label{pchipendpoint}
{k_{1,end}} = \frac{{\left( {2{h_{1*}} + {h_{2*}}} \right){\nu _{1*}} - {h_{1*}}{\nu _{2*}}}}{{{h_{1*}} + {h_{2*}}}}
\end{equation}

The specific rules to give $k_{n}$ above are designed to preserve monotony and avoid overshooting, as the interpolated EOSs are generally expected to be not oscillatory and baratropic.


\end{document}
