\documentclass[UTF8,12pt,a4paper,floats,amsmath,amssymb]{ctexart}

\usepackage{amsmath,amssymb,amsfonts}

\begin{document}

\title{GRACE CG OFFSET DETERMINATION BY MAGNETIC TORQUERS DURING THE IN-FLIGHT PHASE}

\date{\today}

\author{翻译: 陈厚源}

\maketitle

\section{仿真结果与分析}

\subsection{仿真参数与初始值}

我们已经研究了三个仿真案例,其中第一例模拟的是在赤道面附近的情形,而且事实上第五章中的验证性模拟就是这种情况。第二例和第三例分别模拟的是在北极附近以及北极-赤道面之间的情形。图6.1展示了在与这三次模拟对应的位置上,两颗GRACE卫星的磁通密度,需要注意的是这里的坐标展开选取了与卫星固连的参考系。

三次模拟使用的参数(详细取值见第五章)基本上相同,其主要的区别在轨道-姿态参数的初始值以及磁矩。例一中的轨道-姿态参数的初始值可以参照前面的5.2节,而例二和例三的初始值则在下面给出。

%注意这里可能有翻译错误,原文的magnetic moments很可能是一个产生磁力矩的装置(magnetic torque device)而非磁力矩,文中可能混用了磁力矩器和磁力矩。

在案例二中,GRACE前星轨道-姿态动力学方程(2-31)以及活动磁矩(magnetic moments activated)的初始值为:
\begin{equation} \label{attitude_dynamics_2_front}
\begin{array}{l}
\vec{r}(t_0) = [53166.9012866642, 50.3762086975, 6838048.9781353170]\\
\vec{v}(t_0) = [-7627.3588408146, -0.0096034177, 63.3233882824]\\
\vec{q}(t_0) = [-0.9999051930, -0.0000000609, 0.0137697130, 0.0000043659]\\
\vec{\omega}(t_0) = [0.000000000E+00, 0.111490159E-02, 0.000000000E+00]\\
m = {[0, 1, −1]}^{T} \, 30 sin(2 \pi f_0 t)A \cdot m^2
\end{array}
\end{equation}
与之对应的GRACE后星初始值为:
\begin{equation} \label{attitude_dynamics_2_back}
\begin{array}{l}
\vec{r}(t_0) = [291680.8661387600, -48.5674705798, 6831570.5338286750]\\
\vec{v}(t_0) = [-7621.0475706862, -0.0144524709, 329.1764019231]\\
\vec{q}(t_0) = [0.0000000601, -0.9999052485, -0.0000043660, 0.0137656809]\\
\vec{\omega}(t_0) = [0.000000000E+00, -0.111490736E-02, 0.000000000E+00]\\
m = {[0, 1, 1]}^{T} \, 30 sin(2 \pi f_0 t)A \cdot m^2
\end{array}
\end{equation}

在案例三中,GRACE前星轨道-姿态动力学方程(2-31)以及活动磁矩的初始值为:
\begin{equation} \label{attitude_dynamics_3_front}
\begin{array}{l}
\vec{r}(t_0) = [-6739303.8283807930, 48.9237966735, 1198310.5772177500]\\
\vec{v}(t_0) = [-1337.7622725168, 0.1452268716, -7512.2443788538]\\
\vec{q}(t_0) = [-0.7728330709, 0.0000027703, -0.6346093637, 0.0000033749]\\
\vec{\omega}(t_0) = [0.000000000E+00, 0.111490159E-02, 0.000000000E+00]\\
m = {[0, 1, 1]}^{T} \, 30 sin(2 \pi f_0 t)A \cdot m^2
\end{array}
\end{equation}
与之对应的GRACE后星初始值为:
\begin{equation} \label{attitude_dynamics_3_back}
\begin{array}{l}
\vec{r}(t_0) = [-6693453.2487186490, 45.0419246185, 1431899.2109399360]\\
\vec{v}(t_0) = [-1598.8879354469, 0.1417855019, -7461.0147865065]\\
\vec{q}(t_0) = [-0.0000027710, -0.7728282124, -0.0000033745, -0.6346152804]\\
\vec{\omega}(t_0) = [0.000000000E+00, -0.111490736E-02, 0.000000000E+00]\\
m = {[0, -1, 1]}^{T} \, 30 sin(2 \pi f_0 t)A \cdot m^2
\end{array}
\end{equation}

\subsection{仿真结果与分析}
由于两颗GRACE卫星的质心标定(CG offset calibration)互不影响,标定过程既可以同时进行也可以分开独立进行。为了简化模拟,本报告假定两颗卫星的质心标定同时进行。为了检验前面提到的一些标定方法的准确性,GRACE双星的质心偏移量已经在不同情形下被模拟过很多次,具体参数取值参见5.2节和6.1节。

按照地球磁场模型,所有三种情况下都能获得外部磁场,如图6.1所示,而对应的磁矩则可以按照4.2节的方法加载。

GRACE前星的质心偏移量被定为$d = {[0.3,0.4,-0.5]}^{T} mm$,而后星的质心偏移则稍有不同地被定为$d = {[0.3,-0.4,0.4]}^{T} mm$.给定不同的质心偏移,运行整个程序后得出的结果会有所不同,如图6.2和图6.3所示。计算得到的质心偏移量相对于其真实值的均方根偏差已经列在表6.1和表6.2中,两个表格分别对应于GRACE前星和GRACE后星。结合图6.2,图6.3,表6.1以及表6.2的仿真结果,不难看出质心标定的精度会因为标定算法以及进行标定时卫星所在空天位置的不同而不同。实际上,标定精度是与磁矩激活时卫星的角加速度高度相关。在一个方向上的角加速度越大,对应该方向上的质心标定精度就越差。这个结论也是很自然的,毕竟角加速度和质心偏移是密切关联的。

磁矩沿着y轴和z轴方向作用时,在标称相位下的角加速度值如图6.4所示。从图上可以看出,在第一例仿真中,x方向的角加速度比y-z方向的大,这导致了x方向的标定精度要比y-z方向的要差;在第二例仿真中,x方向的角加速度比y-z方向的要大得多,实际上这一例仿真的整个公转周期里x方向的角加速度几乎是最大的,因而x方向的质心标定精度更差;在第三例仿真中,GRACE前星x方向的角加速度比y-z方向的要小,因此这里的x方向标定精度是所有模拟中最高的,而GRACE后星的情况是x方向角加速度依然比y-z方向大,所以后星的x方向标定精度不及其y-z方向。

综上所述,如果磁力矩器启动,在一个轨道周期内进行x方向质心标定的最佳时机是沿着x方向的角加速度比y-z方向小的时候,而进行y-z方向标定的最佳时期则在x方向角加速度比y-z方向大的时候,详见图6.4。看图可知,如果磁力矩器启动,一个公转周期内的大多数时刻角加速度的x分量都会比其y-z分量要大,只有在少数时段(持续长度大约100秒)才会出现x方向角加速度较y-z方向小的情况。因此,本文认为质心标定可以分两步进行,即分别选取合适的时刻对x方向和y-z方向分开进行标定以取得更高的标定精度。从表6.1和表6.2可以看出,x方向的质心标定精度可以达到0.02mm,而y方向和z方向的精度则分别可以达到0.01mm和0.02mm。值得注意的是,这些指标都是按照GRACE前星在各个预设的最佳时机对不同方向分开进行标定而得到的结果。

\end{document}
